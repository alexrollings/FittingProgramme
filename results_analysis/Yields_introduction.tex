\documentclass[12pt, landscape]{article}
\usepackage[margin=0.1in]{geometry}
\usepackage{mathtools}
\usepackage{float}
\usepackage[table]{xcolor}
\usepackage{xcolor}
\restylefloat{table}
\begin{document}

\textbf{Yield definitions in the FAV mode} (\colorbox{blue!25}{Floating RRV},
\colorbox{green!35}{Gaussian constrainted RRV}, \colorbox{pink}{Fixed
RRV}):

\begin{itemize}
  \item Yields for signal decays combined with the wrong neutral particle (WN)
  are defined in each category, in relation to the signal yield, as:
  \begin{equation*}
    \colorbox{blue!25}{\text{N(WN)}} = \colorbox{blue!25}{\text{N(signal)}} \times
    \colorbox{pink}{$\frac{\epsilon_{cross}\text{(WN)}}{\epsilon_{cross}\text{(signal)}}$}
    \times
    \colorbox{pink}{$\frac{\epsilon_{sel}\text{(WN)}}{\epsilon_{sel}\text{(signal)}}$}
    \times 
    \colorbox{green!35}{$f_{WN}$}
  \end{equation*}
  Where $f_{WN}$ is a floating RRV, representing the relative efficiency of
  combining with the wrong neutral in data, compared to MC, and is shared across
  both $D^*\rightarrow D\gamma$ and $D^*\rightarrow D\pi^0$ components. $f_{WN}$
  is a gaussian constrained quantity, $\mu=1$, $\sigma=0.1$.

  In the $D\pi^0$ fit, the $D\gamma$ WN yield is only calculated this way in the
  favoured mode, where it is also corrected for by the branching ratio of the
  two $D^*$ decays. In the other categories, the yields are related to the FAV
  yield by physics parameters.  \item $B^0\rightarrow D^{*\pm}h^{\mp}$,
  $B^{\pm}\rightarrow Dh^{*\pm}$ and $B^{\pm}\rightarrow D^{*0}h^{*\pm}$ MC
  samples have the same simulation versions, therefore any effect on the
  efficiencies from the mis-modelling of the underlying event in the samples
  should be shared.
  \item The $B^0\rightarrow D^{*\pm}\pi^{\mp}$ yield is a floating RRV.
  \item The $B^{\pm}\rightarrow D\rho^{\pm}$ yield is a gaussian constrained RRV:
    \begin{equation*}
      \colorbox{green!35}{\text{N($B^{\pm}\rightarrow D\rho^{\pm}$)}} =
      \colorbox{blue!25}{\text{N($B^0\rightarrow D^{*\pm}\pi^{\mp}$)}} \times 
      \colorbox{pink}{$\frac{BF(B^{\pm}\rightarrow
      D\rho^{\pm})}{BF(B^0\rightarrow D^{*\pm}\pi^{\mp})}$} \times
      \colorbox{pink}{$\frac{\epsilon_{sel}(B^{\pm}\rightarrow
      D\rho^{\pm})}{\epsilon_{sel}(B^0\rightarrow D^{*\pm}\pi^{\mp})}$}
    \end{equation*}
    The central value of the gaussian is the result of the above calculation,
    and the standard deviation is the combined error of the selection
    efficiencies and branching fractions.
  \item The $B^{\pm}\rightarrow (D^*\rightarrow D\gamma(\pi^0))\rho^{\pm}$
    yields are gaussian constrained RRVs. In the $D\gamma(D\pi^0)$ fit, the
    $D\gamma(D\pi^0)$ component of this decay is constrained relative to the
    $B^0\rightarrow D^{*\pm}\pi^{\mp}$ yield:
    \begin{equation*}
      \colorbox{green!35}{\text{N($B^{\pm}\rightarrow (D^*\rightarrow
      D\gamma(\pi^0))\rho^{\pm}$)}} =
      \colorbox{blue!25}{\text{N($B^0\rightarrow D^{*\pm}\pi^{\mp}$)}} \times 
      \colorbox{pink}{$\frac{BF(B^{\pm}\rightarrow (D^*\rightarrow
      D\gamma(\pi^0))\rho^{\pm})}{BF(B^0\rightarrow D^{*\pm}\pi^{\mp})}$} \times
      \colorbox{pink}{$\frac{\epsilon_{sel}(B^{\pm}\rightarrow (D^*\rightarrow
      D\gamma(\pi^0))\rho^{\pm})}{\epsilon_{sel}(B^0\rightarrow
      D^{*\pm}\pi^{\mp})}$}
    \end{equation*}
    In the $D\gamma(D\pi^0)$ fit, the $D\pi^0(D\gamma)$ component of this decay
  is constrained relative to the $D\gamma(D\pi^0)$ component (calculated above):
    \begin{equation*}
      \colorbox{green!35}{\text{N($B^{\pm}\rightarrow (D^*\rightarrow
      D\pi^0(D\gamma))\rho^{\pm}$)}} =
      \colorbox{green!35}{\text{N($B^{\pm}\rightarrow (D^*\rightarrow
      D\gamma(\pi^0))\rho^{\pm}$)}} \times 
      \colorbox{pink}{$\frac{BF(D^*\rightarrow
      D\pi^0(D\gamma))}{BF(D^*\rightarrow D\gamma(\pi^0))}$} \times
      \colorbox{pink}{$\frac{\epsilon_{sel}(B^{\pm}\rightarrow (D^*\rightarrow
      D\pi^0(D\gamma))\rho^{\pm})}{\epsilon_{sel}(B^{\pm}\rightarrow
      (D^*\rightarrow D\gamma(\pi^0))\rho^{\pm})}$}
    \end{equation*}
    Again, the central values of the gaussians are the result of the
    calculations, and the standard deviations are the combined error of the
    selection efficiencies and branching fractions.
  \item $R(D^*K/D^*\pi)$ for $B^0\rightarrow D^{*\pm}h^{\mp}$ and
    $B^{\pm}\rightarrow Dh^{*\pm}$ are fixed; $R(D^*K/D^*\pi)$ for
    $B^{\pm}\rightarrow D^*h^{*\pm}$ is a gaussian constrained RRV, with a
    standard deviation of 10\% (essentially floating).
\end{itemize}

\end{document}
